\newpage
\section*{
\vskip 1em
ABINIT : 7. EASY TO LEARN AND USE}

\vskip 1em
\parbox{0.92\linewidth}{
\Large{
How difficult is it to learn to use ABINIT, \\
and then, to use it for daily work ?

\vskip 2em
The concept of {\blue self-learning}
is central to the ABINIT project : \\
at his fingertips,
the user will find all the information he/she needs to learn
to use ABINIT
\vskip 1em
\begin{itemize}
 \item[\red \ding{229}]
  Tutorials : four {\green basic tutorials} of 2 hours each (used to teach classes), and more than thirty {\green specialised tutorials}
 \item[\red \ding{229}]
  On-line help files, and on-line list of input variables
 \item[\red \ding{229}]
  The input files used in the automatic tests show how
  to use {\blue all} the functionalities of ABINIT (more than 500 files).
 \item[\red \ding{229}]
  ... no need to attend a school !
\end{itemize}
\vskip 1em
The daily use has also been at the center of development efforts :
\vskip 1em
\begin{itemize}
 \item[\red \ding{229}]
  Pseudopotentials are available for all the elements, in LDA,
  or can be generated for the GGA
 \item[\red \ding{229}]
  Input by keywords and default values
 \item[\red \ding{229}]
  Metalanguage for chaining different datasets
 \item[\red \ding{229}]
  Automatic renaming of output files for bookkeeping
\end{itemize}
}
}
