\documentclass[11pt]{article}
\pagestyle{empty}
\textheight=26cm
\topmargin=-2cm
\textwidth=17cm
\oddsidemargin=-0.5cm
\begin{document}
{\Huge \bf Help for the AIM utility of the ABINIT package} \\ \\

     Copyright (C) 2002-2018 ABINIT group (PCasek,XG)
     This file is distributed under the terms of the
     GNU General Public License, see ~abinit/COPYING
     or http://www.gnu.org/copyleft/gpl.txt .
     For the initials of contributors, see ~abinit/doc/developers/contributors.txt .

The AIM utility allows to analyse charge densities
produced by the ABINIT code. The AIM analysis (Atom-In-Molecule)
has been proposed by Bader. Thanks to topological
properties of the charge density, the space is partitioned
in non-overlapping regions, each containing a nucleus.
The charge density of each region is attributed to the
corresponding nucleus, hence the concept of Atom-In-Molecule.
References to the relevant litterature are to be provided.

The aim utility is generated in the same way as other
ABINIT utilities (i.e. make allseq).

To run the program one needs to prepare two files:
\begin{description}
\item [files-file] file which contains the name of the input files and the
root for the names of the different output files.
\item [input-file] file which gives the values of the input variables
\end{description}
Except this file you need the output file of ABINIT with the valence
density in real space $(*\_DEN)$ and the core density files $(*.fc)$ sorting
from the program for generating pseudopotentials FHI.
LDA core density files have been generated for the
whole periodic table, and are available on the ABINIT web site.
Since the densities are weakly dependent on the choice of the
XC functional, and moreover, charge density analysis
is mostly a qualitative tool, these files can be used
for other functionals. Still, if really accurate
values for the Bader charge analysis are needed, one should
generate core density files with the same XC functional as
for the valence density.

Usually, the grid in the real space for the valence density should be
taken more fine than the one proposed by ABINIT.
(For example for the lattice parameter 7-8~a.u. , ngfft at least 64
gives the precision of the Bader charge estimated to be better
than 0.003~electrons).
To run the program type: $$aim\; < \; <files.file> $$

The structure of the {\bf files-file} is as follows: \\[1.5\baselineskip]
\begin{tabular}{cl}
 *.in   & \# input-file \\
 *.den  & \# valence density (output of ABINIT) \\
 $<root>$ & \# the root of the names of different output files \\
 at1.fc & \# core density files (in the same order as \\
 ...    & \# in the ABINIT files-file) \\
 atN.fc & \\
\end{tabular} \\[1.5\baselineskip]
The example of the files-file is shown in the file {\it ./example/mgo.files}.\\

The {\bf input-file} is made in the same manner as the input file of
ABINIT - it suffices to give the name of the input variable following
by the value(s) separated by one or more white character(s).
The names are not case sensitive. All input variables have the meaningful
default value, however, you have to specify what you want to calculate
(default = nothing). The example of the simple input file for Oxygen in
bulk MgO is in {\it ./example/mgo.in}. There are also the
corresponding output files in this directory.\\

{\large \bf \centering The output files}\\ \\
The atomic units and cartesian coordinates are used for all output
parameters. The names of the output files are of the form $<root>.suffixe$.
There are three main output files: \\
\begin{description}
\item [*.crit] - the file with the CPs - they are listing in the order BCPs, RCPs and CCPs.
The line of the output contains these informations:
$$ position \quad \frac{eigen\  values}{of\  Hessian} \quad \frac{index\
of\  the}{bonded\  atom}\footnote{BCPs only} \quad \Delta \rho_c \quad \rho_c, $$
where position is done with respect to the considered atom.
\pagebreak
\item [*.surf] - the file with the Bader surface - there is the head
of the form:\\[3mm]
\begin{tabular}{ccc}
index of the atom & \multicolumn{2}{c}{position} \\
ntheta & thetamin & thetamax \\
nphi & phimin & phimax \\
\end{tabular} \\
and the list of the Bader surface radiuses:
$$ \theta \quad \phi \quad r(\theta,\phi) \quad
W(\theta,\phi)\footnote{the weight for the Gauss quadrature}. $$
The minimal and maximal radiuses are done at the last line.
\item [*.out] - the central output file - there are many informations
which are clearly described.
Here is some additional information on the integration - there is separated
integrations of the core and the valence density.
In both cases the radial integration
is performed using cubic splines and the angular ones by Gauss
quadrature. However the principal part of the core density
{\it of the considered atom} is integrated in the
sphere of minimal Bader radius using spherical symmetry.
The rest of the core density (of this atom, out of this sphere)
together with all the core contributions of the neighbors are added to the
valence density integration. In the ouput file, there is the result of
the {\it complete} integration of the core density of the atom,
then the two contributions (spherical integration vs. the others)
and then the total Bader charge.
\item [*.log] - the log file - a lot of informations but it is not
very nice actually.
\item [*.gp] - gnuplot script showing the calculated part of the Bader
  surface with lines.
\end{description}
The gnuplot scripts are made in the manner that one needs type only\\
{\bf load  '{\it file}'} (quotes are necessary).\\
Note, that this isn't considered as
the visualization (it is only for working purpose)!



{\large \bf \centering The list of the input variables}\\ \\
The atomic units and cartesian coordinates are used for all input
parameters. \\

{\bf \centering  Driver variables}\\
\begin{description}
\item [surf] the calculus of the Bader surface
  \begin{description}
    \item [0(d)] not
    \item [-1] reading from the file ``root''.surf
    \item [1] calculated
  \end{description}
\item [crit] the calculus of the critical points
  \begin{description}
    \item [0(d)] not
    \item [-1] reading from the file ``root''.crit
    \item [1] calculated (simplified version)
    \item [2] calculated (standard version)
    \item [3] calculated (the original version)
  \end{description}
  {\bf The original version} searches all CPs starting from the center betwen
  two and three atoms (atom - neighbor(s)) by
  Newton-Raphson algorithm - without tests (not recommended) - don't
  use together with surface analyse!
{\bf The simplified and standard version} searches CP(3,-1) starting
  from the center of the pairs~atom-neighbor; then CP(3,1) from the
  center between two CP(3,-1) and finally CP(3,3) from the center
  between two CP(3,1). The robust Popeliers's algorithm is used.
  The difference between two is based in the fact
  that {\bf the standard version} makes the test if the CP is really
  on the Bader surface of the calculated atom for each CP, while
  {\bf the simplified version} only for CP(3,-1). When CP analyse is
  rather fast (with respect surface determination) 2 is recommended.
  In all cases the number of neighbors considered is limited by
  distance cutoff (variable {\bf maxatd})
\item [irho] the integration of the charge of the Bader atom
  \begin{description}
    \item [0(d)] not calculated
    \item [1] calculated
  \end{description}
\item [ivol] the integration of the volume of the Bader atom
  \begin{description}
    \item [0(d)] not calculated
    \item [1] calculated
  \end{description}
\item [rsur] the determination of the rayon of the Bader surface for
the angles specified in the input variable {\bf rsurdir}.
  \begin{description}
    \item [0(d)] not calculated
    \item [1] calculated
  \end{description}
\item [follow] follow the gradient path to the corresponding atom
starting from the position specified in the input variable {\bf foldep}.
  \begin{description}
    \item [0(d)] not calculated
    \item [1] calculated
  \end{description}
\item [denout,lapout] the additional output of electronic density
and of the laplacian of electronic density. The specification of the
line (plane) in the real space must be given in the input variable
{\bf vpts} and grid in {\bf ngrid}. It is also possible to get only the
valence density or the core density (see {\bf dltyp}).
  \begin{description}
    \item [0(d)] not
    \item [1] 1D distribution
    \item [2] 2D distribution
  \end{description}
\item [dltyp] the specification of the contribution of the electronic
density corresponding to the additional output (denout, lapout).
  \begin{description}
    \item [0(d)] total electronic density
    \item [1] only the valence density
    \item [2] only the core density
  \end{description}
\item [gpsurf]  the additional graphic output (gnuplot script) of the
irreducible part of the calculated Bader surface.
  \begin{description}
    \item [0(d)] not
    \item [1] given
  \end{description}
\end{description}

{\bf Input variables}\\
\begin{description}
\item [atom] the index of the investigated atom (1 - default)
\item [nsa,nsb,nsc] the number of the primitive cell to be constructed
in each direction and each sense of the primitive translation to
simulate lattice ((3,3,3) - default).
\item [inpt] the number of radial points used for integration of the
Bader charge (100 - default) (not too sensitive).
\item [ntheta,nphi] the angular grid for the integration of the Bader
charge (32 $\times$ 48 - default). Proposition : 20 for $[0,\pi/2]$,
32~for $[0,\pi]$ and 48 for $[0,2\pi]$.
\item [thetamin,thetamax] the limit of the angular region of the
Bader surface which is calculated ($[0,\pi]$ - default).
\item [phimin,phimax] the limit of the angular region of the
Bader surface which is calculated ($[0,2\pi]$ - default)
\item [atrad] the first estimation of the Bader radius (not too
important - it is used only two times) (1.0 - default).
\item [radstp] the first step in searching the exact Bader radius
(0.05 - default) (0.03 is well usually)
\item [folstp] the first step for following the gradient path
(0.05 - default)
\item [ratmin] the first estimation of the least radius of the bassin of the
atom (the distance in which the procedure following the gradient path
annonces that the gradient path finishes in the corresponding atom)
(1.0 - default). This parameter is very important for the speed of the
calcul, but this first estimation is not usually used because program
makes the new one based on the knowledge of CPs. In fact after
the CP analyse, the new estimation is done by the product of adhoc
parameter coff1 (0.98) ({\bf adhoc.f90}) and the distance of the nearest
bonding CP. If there is problem later, coff2 (0.95) is used instead.
\item [scal] the scaling of the cartesian coordinates for the calcul
of the distances ($x'[i]=x[i]/scal[i]$)
($[1.0,1.0,1.0]$ - default) - not really usefull.
\item [maxatd] - the maximal distance for the atoms considering for
searching CP (10.0 - default).
\item [maxcpd] - the maximal distance of CPs considering for the
corresponding atom (5.0 - default).
\item [rsurdir] variable only for the case rsurf=1 i.e. for the
  calcul of the singular Bader radius (rsur) = corresponding
  angular coordinates ($[0.0,0.0]$ - default).
\item [foldep] variable only for the case follow=1 i.e. for the
procedure following only the gradient path (follow) - starting point
($[0.0,0.0,0.0]$ - default).
\item [ngrid] dependent variable for the additional output
(denout, lapout) - the grid in the real space ($[30,30]$ - default)
\item [vpts] dependent variable for the additional output
(denout, lapout) - the points defining the basic vector(s) of the line
or rectangle in real space (the first = the basic point)
(all points are par default nulify).
\end{description}
\end{document}
