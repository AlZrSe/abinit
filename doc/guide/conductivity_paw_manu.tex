%\documentclass[superscriptaddress]{revtex4-1}
\documentclass{article}

\begin{document}
%
\title{Conductivity calculations using the PAW formalism }

\author{S. Mazevet, V. Recoules, M. Torrent, et F. Jollet,
{\it D\'epartement de Physique Th\'eorique et Appliqu\'ee,
CEA/DAM \^Ile-de-France,
BP12, 91680 Bruy\`eres-le-Ch\^atel Cedex, France}}

\date{\today}

\vspace{0.5cm}
To perform a conductivity calculation within the PAW formalism you need to first use
a PAW potential and run a ground state calculation with the \textit{prtnabla} variable
 set to 1 and \textit{prtwfk=1}.
This calculates the necessary matrix elements and creates a file named filename \_OPT.

The postprocessor \textit{conducti} read the file filename \_OPT and calculate
the electrical and thermal conductivity.

\textit{conducti < filename.files}

where \textit{filename.files} contains the input and output filenames.
\vspace{0.25cm}

\textit{filename.in} contains the following variables in the PAW case:


2            ! 2 for PAW calculations


filename     ! generic name of the ground state data files obtained with   prtwfk=1


0.073119 0.0000001 5.00 1000   !gaussian width, omega\_min, omega\_max, nbr pts
\vspace{0.5cm}

Warning the conducti input file is for the moment different when used in the
PAW and NCPP modes. With NCPP, the input file is (see \textit{/doc/users/conducti\_manuel.tex})




1                 ! 1 for norm-conserving calculations


t78o\_DS3\_1WF4 ! 1st DDK file


t78o\_DS4\_1WF5 ! 2nd DDK file


t78o\_DS5\_1WF6 ! 3rd DDK file


t78o\_DS2\_WFK  ! ground state data file obtained with   prtwfk=1


9.50049E-04   ! temperature


1.000         ! k point weigth


0.00735  2.0  ! Gaussian and frequency width; omega-max

\end{document}
