\documentclass[a4,12pts]{article}

\usepackage{graphicx}
\usepackage{bm}

\begin{document}

\title{Optical conductivity using abinit with norm-conserving pseudopotentials}
\author{V. Recoules}

\maketitle

\section{Kubo-Greewood formula implementation}
The real part of the optical conductivity versus frequency $\omega$ is
computed using the
Kubo-Greenwood formulation \cite{KUBO57, GREE58} and can be expressed as :
%
\begin {eqnarray}
\sigma(\omega) =&&{\frac {2\pi} {3 }}{\frac 1 \Omega}\sum_{\bm k} W({\bm k})
 \sum_{n,m}(f_n^{\bm{k}}-f_m^{\bm{k}}) \nonumber \\
&&{\frac 1 {(2 \pi)^2}}\times \left | \left < \psi_n^{\bm{k}} \left |
\frac{\partial \hat{H}}{\partial \bm{k}} \right|\psi_m^{\bm{k}}\right >
  \right|^2 \delta(E_m^{\bm{k}}-E_n^{\bm{k}}- \omega)
 \label{KUGR}
\end {eqnarray}
%
where
\begin{itemize}
\item $\Omega$ is the volume of the unit cell,
\item $W(\bm{k})$ is the $\bm{k}$-point weight in the Monkhorst-Pack scheme,
\item $f_n^{\bm{k}}$ is the Fermi-Dirac occupations for the band n and the k-point $\bm{k}$,
\item the matrix element $\left < \psi_n^{\bm{k}} \left |
\frac{\partial \hat{H}}{\partial \bm{k}} \right|\psi_m^{\bm{k}}\right >$
 are  computed using a response function calculation.
Technical details on the computation of matrix
elements can be found in Refs.  \cite{GONZ97, GONZL97}.
\item The $\delta$ function can be resolved by averaging over a finite frequency interval
$\Delta \omega$ \cite{PFAF97}:
%
\begin{equation}
 \sigma(\omega_l) = \frac{1}{\Delta \omega} \int_{\omega_l-\frac{\Delta \omega}{2}}
  ^{\omega_l+\frac{\Delta \omega}{2}} \sigma(\omega) d\omega
 \label{AVER}
\end{equation}
This was done until version 4.0.2, and now, a Gaussian is used to represent the delta function:
%
\begin{equation}
G(\omega)=\frac{1}{\sqrt{\pi}\Delta}exp\left[ -((E_m^{\bm{k}}-E_n^{\bm{k}}-\hbar \omega)/\Delta)^2
\right]
\end{equation}
%
\end{itemize}
%




\section{Abinit computation}
An example of input file for abinit is given in Test\_v3/t78.in. Five linked
calculations have to be run to obtained all the needed quantities to evaluate
the optical conductivity.
\begin{itemize}
\item First calculation is a scf evaluation of the electronic density, with iscf=3 and
prtden=1. The Fermi-Dirac occupations is obtained at this step.
\item Second calculation is a non-scf calculation (iscf=-2) starting from the previous electronic density
and wave functions. It gives the Khon-Sham eigenvalues needed.
\item The three last calculations are calculations of response function (iscf=-3) and
gives the derivative of the Hamiltonian with respect to the wave vector
for the three directions. There is one calculation for each direction.
They start from the wave functions of the previous non-scf calculation (getwfk=2).

\end{itemize}


\section{conducti module}
An example of input file for conducti is given in Test\_v3/t79.in.
\begin{verbatim}
1                 ! 1 for norm-conserving calculations
t78o_DS3_1WF4
t78o_DS4_1WF5
t78o_DS5_1WF6
t78o_DS2_WFK
9.5E-04
1.00000
0.00735 2.0
\end{verbatim}

\begin {itemize}
\item the three first lines are the name of the xxx\_1WFxx function for the three directions,
\item  then there is the
name of the wave function file obtained with the non-scf calculation.
\item The temperature is given in a.u.
\item The weight of the k-vector is given exactly with the same format as the output file.
\item The first number is the value of $\Delta$ in a.u., the conductivity is computed up to the second number.
\end{itemize}.


When conducti runs, you will have to answer with the name of the input file containing all the
above informations and the name of the output file.

\section{Output content}
First, you get some information (as rprimd, the number of bands, of k-points, etc..) to be
sure that the reading is okay.

The sum rule is computed \cite{KUBO57}:
%
\begin{equation}
 S= \frac{2 m_e \Omega}{\pi e^2 n_e} \int_0^\infty \sigma(\omega) d\omega
\end{equation}
%
where $m_e$ is the electron mass and $n_e$ the electronic density. S must be equal to 1 (or
as close as possible).
To obtain a well converged sum rule, more unoccupied electronic
states is needed than for the
determination of the DC electrical conductivity $\sigma(\omega \rightarrow 0)$.

At last, since a finite number of excited states is included in the calculation,
$\sigma(\omega)$ is computed correctly for $\hbar\omega<|E_{\rm max}-E_{\rm f}|$ where
$E_{\rm f}$ is the Fermi level and $E_{\rm max}$ is the energy of the highest level computed.
This value is computed.


The output file contain the energy value in Ha and eV and the optical conductivity in
a.u and (ohm.cm)$^{-1}$.

The DC electrical conductivity $\sigma_{{\rm DC}}$ computation is made by
extrapolating optical conductivity to $\omega=0$. An example
of application is given in \cite{RECO02}.

Conducti produces different files, with the following brief description :
(1) the zero limit of *.STP is the thermopower ; (2) *.SIG is the real part of the
optical conductivity, the zero limit gives the electrical conductivity ;
(3) the zero limit of *.KTH gives the thermal conductivity ;
(4) *.TENS is the electrical tensor including the non-diegonal terms.

\begin{thebibliography}{1}

\bibitem{KUBO57}
R.~Kubo.
\newblock Statistichal -mechanical theory of irreversible precesses. i.
\newblock {\em Journal of the Physical Society of Japan}, \textbf{12}, 570 (1957).

\bibitem{GREE58}
D.~A. Grenwood.
\newblock The boltzmann equation in the theory of electrical conduction in
  metals.
\newblock {\em Proc. Phys. Soc.}, \textbf{715}, 585 (1958).

\bibitem{GONZ97}
X.~Gonze.
\newblock First-principles responses of solids to atomic displacements and
  homogenous electric fiels : implementation of a conjugate-gradient algorithm.
\newblock {\em Phys. Rev. B}, \textbf{55}, 10337 (1997).

\bibitem{GONZL97}
X.~Gonze and C.~Lee.
\newblock Dynamicals matrices, born effective charges, dielectric permittivity
  tensors and interatomics force constants from density functional perturbation
  theory.
\newblock {\em Phys. Rev. B}, \textbf{55}, 10355 (1997).

\bibitem{PFAF97}
O.~Pfaffenzeller and D.~Hohl.
\newblock Structure and electrical conductivity in fluid-density hydrogen.
\newblock {\em J. Phys.:Cond. Matter.}, \textbf{9}, 11023 (1997).

\bibitem{RECO02}
V.~Recoules, P.~Renaudin, J.~Cl{\'e}rouin, P.~Noiret, and G.~Z{\'e}rah.
\newblock Electrical conductivity of hot expanded aluminum: Experimental
  measurements and {\it ab initio} calculations.
\newblock {\em Phys. Rev. E}, \textbf{66}, 056412 (2002).

\end{thebibliography}

\end{document}
